\section{Overview}

\subsection{Objectives}

\subsection{Outline}

\subsection{Background}

This lesson relies heavily on the previous lessons, so we summarize the relevant content that they have worked on already prior to this one:
\begin{itemize}
    \item Students have been working with polynomials in general for a while now: adding, subtracting, multiplying, dividing them, and finding their zeros.
    \item Specifically, students found that polynomials can be written in the form $P(x) = (x-r_1)(x-r_2)\cdots (x-r_n)$, where $r_1$, $r_2$, $\dots$, $r_n$ are the roots of $P$.
    \item Students learned the term \textit{degree} for a polynomial, and may have noticed already that a polynomial has as many roots as its degree.
    \item Students learned that if $Q(r) = 0$ and $Q(x) \mid P(x)$ for polynomials $P$ and $Q$, then $P(r) = 0$ as well. In particular, if $r$ is a root of $P$, then $P(x) = (x-r)T(x)$, where $T$ is a polynomial and $\deg T = \deg P - 1$.
    \item Students have worked with complex numbers in the past with quadratics.
\end{itemize}

\subsection{Materials}

\section{Content}