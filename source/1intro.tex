\chapter{Introduction to Polynomials}

\emph{By Mia Petrie}

\section{Overview}

\subsection{Objectives}

\begin{outline}
    \1 Students should be able to recite key terms and concepts.
    \1 Students will be able to demonstrate this knowledge by breaking down the parts of a polynomial.
    \1 Students will apply their knowledge of polynomials by successfully adding and subtracting polynomials.
\end{outline}

\subsection{Materials}

Each student will need to have a pencil and paper to take notes and complete worksheets given to them. They will also need some way to connect to online games, whether that be through a Chromebook or their own personal phone. I will need some way of presenting a slideshow (e.g., Computer and projector of sorts), and enough printed-off worksheets for the whole class.

\section{Outline}


Total lesson length: approx. 50 minutes

\textbf{Adding and Subtracting Polynomials}

\begin{outline}
    \1 Explanation of today’s lesson, and present the learning objectives (Slide 2)
    \1 Review Game (5 minutes), top winners will get candy (Slide 3)
        \2 This game provides an \textbf{incentive} for students to pay attention and learn the terms
    \1 Further explanation of terms/labeling parts of a polynomial (5 minutes) (Slides 4, 5, \& 6)
        \2 Vocab: Terms, coefficients, exponents, variables, degree, and constants.
        \2 What makes a polynomial?
    \1 Give examples of polynomials and non-polynomials. (5 minutes) (Slide 7)
        \2 3 Not polynomials, identify why.
        \2 3 real polynomials, identify correctly.
    \1 Standard Form (5 minutes) (Slides 8 \& 9)
        \2 Getting polynomials into standard form 
            \3 Give 3 examples; 2 rearranging and 1 distributing.
    \1 Explain the process of adding Polynomials (10 minutes) (Slides 10 \& 11)
        \2 Like terms (look at exponents)
        \2 Be aware of signs (positive or negative)
        \2 Do 3 examples of varying difficulty:
            \3 Adding, fairly simple, basics
            \3 Longer, more negative numbers
            \3 Long, degree 7
    \1 Explain the process of subtracing polynomials
        \2 Like terms
        \2 Carrying over the subtraction to everything
        \2 Do 2 examples of varying difficulty
            \3 Short and sweet, intro carrying
            \3 Higher degree, more complex
        \2 Showing these examples and walking them through problems helps give them the \textbf{scaffolding} they need until they are able to complete them on their own.
            \3 This lesson was mostly the \textbf{direct-instruction approach} where the teacher lectures to give the students new information.
    \1 Hand out homework worksheets. Students are free to work in groups or pairs. Encourage them to try to figure it out with their peers without teacher intervention. Will be graded on effort/completion and gone over next class. They have the remaining class time to work on it.
        \2 Homework helps with \textbf{rehearsal}, ensuring they practice the content until they can do it easily and know the processes.
\end{outline}

\href{https://kutasoftware.com/FreeWorksheets/Alg1Worksheets/Adding+Subtracting%20Polynomials.pdf}{Worksheet Link}

\href{https://www.canva.com/design/DAF0A7UZXcA/X2H6gbI5I4nYcRe6sKrcHQ/edit?utm_content=DAF0A7UZXcA&utm_campaign=designshare&utm_medium=link2&utm_source=sharebutton}{Slideshow Link}

\section{Steps}

\begin{enumerate}
    \item We will start out class by going through a list of what will be discussed in class today. This is located on the provided slideshow. This lesson will utilize the \textbf{direct-instruction approach} in order to efficiently introduce the information to the students.
    \newpage

    \item Next, we will play a quick review game that covers key terms related to polynomials. Please have all of the students take out a device that they can play Kahoot on (this could be their school Chromebook or phone, whatever they have access to). These should mostly be review, and not new terms. Playing this game, in the beginning, will remind the students of these terms and force them to \textbf{retrieve} these terms from their memory. The top three winners will get a small piece of candy, this is a type of \textbf{incentive} to encourage students to participate and try their best.
    \item After the game we will return to the slideshow and redefine these terms, \textbf{rehearsal} of the definitions helps students to remember. 
    \item The next slide has a polynomial function. Ask the students to examine it and name each part of it (list the terms, coefficients, exponents, what is the variable, and degree?).
    \item The next slide has the definition of a polynomial and different types of polynomials. Getting a concrete knowledge of the basics before moving on to the rest of the lesson helps to promote \text{mastery learning}.
    \item The next slide provides a list of polynomials and non-polynomials. Ask the students to identify what is and isn't based on the previously provided definition.
    \item The next 5 slides cover the basics of standard form, converting to standard form, adding polynomials, and subtracting polynomials. Walk the students through each problem on the slide and allow them to ask questions when necessary.
    \item The final slide indicates that it is homework time. Pass out the printed-off worksheets. As the slides explain, students are free to work in groups, pairs, or individually. They have the remaining class time to work on it and ask any questions, though encourage them to ask their classmates what they think before going to the teacher.
\end{enumerate}