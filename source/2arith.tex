\chapter{Polynomial Multiplication}

\section{Overview}

\subsection{Objectives}

\begin{outline}
    \1 Students can demonstrate an understanding of how to multiply polynomials.
    \1 Students can check problems with classmates and are able to identify right and wrong answers.
    \1 Students can apply their knowledge to more complex problems.
\end{outline}

\subsection{Materials}

Each student will need to have a pencil and paper to take notes and complete worksheets given to them. I will need some way of presenting a slideshow (ex. Computer and projector of sorts), and enough printed-off worksheets for the whole class.

\section{Outline}

Total lesson length: approx. 50 minutes

\begin{outline}
    \1 Explanation of today's lesson, and present the learning objectives (Slide 2)
    \1 Review Homework and Answer any Questions (10 minutes) (Slide 3)
        \2 This allows for better \textbf{mastery learning} as I am able to tell if the students fully understand the material before continuing the lesson.
    \1 Introduce Multiplying Simple/Longer Polynomials (15 minutes) (Slides 4, 5, \& 6)
        \2 Walk through the process
        \2 Add exponents, combine like terms
        \2 Do 3 examples
            \3 Monomial
            \3 Binomials (with FOIL)
            \3 A Trinomial and a binomial
    \1 Break into pairs, they can choose partners (5-8 minutes) (Slide 7 \& 8)
        \2 Give each group a long polynomial to work through. ($\sim$degree 5 $\times$ degree 5)
        \2 Allow them to ask questions and check each other.
            \3 This is a type of \textbf{cooperative learning} that allows students to share ideas and discuss.
    \1 Have each group pair up with another and walk them through each other's problems. Chnage to the next slide that has the answers to each problem, slide 9, so they can check their work. ($\sim$5-8 minutes)
        \2 Allows for \textbf{peer-to-peer} tutoring as the students can help each other if they had any trouble answering the question, and even if they did not they can explain their thought process and how they completely their problem. It is also a type of \textbf{formative assessment} as it assesses their knowledge during the lesson and by paying attention to what they are discussing I can determine if they have fully understood the material.
    \1 Hand out homework worksheets. Students are free to work in groups or pairs. Encourage them to try to figure it out with their peers without teacher intervention. Will be graded on effort/completion and gone over next class. They have the remaining class time to work on it. (Cooperative Learning)
\end{outline}

\href{LINK}{Worksheet Link}

\href{https://www.canva.com/design/DAF0BNXNsS0/tsfpypAb2t-n2v5nL9Hqkw/edit?utm_content=DAF0BNXNsS0&utm_campaign=designshare&utm_medium=link2&utm_source=sharebutton}{Slideshow Link}

\section{Steps}

\begin{enumerate}
    \item This lesson will start off by going over an agenda for today's lesson. This is located in the provided slideshow.
    \item Next, we will go over the previous lesson's homework. This is time for the students to ask any questions they may have had and the teacher can walk them through any difficult problems. This is a type of \textbf{formative assessment} and allows the teacher to understand if the students understood the previous lesson, and modify the current lesson if needed.
    \item Next, we will introduce a new topic, multiplying polynomial equations. The next 3 slides have examples of problems with increasing difficulties. Walk the students through each problem and answer any questions they have. This provides students with \textbf{scaffolding} as they are introduced to more difficult problems.
    \item The next slide explains a partner activity. Separate the students into pairs and have each pair pick one of the ten problems on the following slide. They will have a few minutes to work through it together. This is a type of \textbf{peer-to-peer tutoring} as they are able to help each other.
    \item Once all of the pairs appear to have completed their problems, or 8 minutes have passed, ask the pairs to meet with other pairs to discuss their problems and explain what they did to the other groups. Provide them with the answers to their problems (located on the following slide) so that if they do not have the correct answer they can find out why with their new group. This is a type of \textbf{cooperative learning} as they are helping each other and bringing together what they have learned, similar to the jigsaw method.
    \item The final slide indicates that it is homework time. Pass out the printed-off worksheets. As the slides explain, students are free to work in groups, pairs, or individually. They have the remaining class time to work on it and ask any questions, though encourage them to ask their classmates what they think before going to the teacher.
\end{enumerate}
