\chapter{Quadratics}

\emph{By Nathan Diaz}

\section{Overview}

\subsection{Objectives}

\begin{itemize}
    \item Students will be able to learn how they could able the quadratic formula to real life situations
    \item Students able to identify coefficients that are associated with variables 
    \item Students able to repeat the formula naturally
\end{itemize}

\subsection{Duration}

\begin{itemize}
    \item 2 minutes - rapid fire of items that represent the a parabola 
    \item 20 minutes - lecturing through the slides with practice problems
    \item 25 minutes - collaboration within small group 
\end{itemize}

\subsection{Materials}

Teachers will need:
\begin{itemize}
    \item White board 
    \item Dry erase marker 
    \item TI-84 calculator 
    \item Slideshow (and a way to display them)
\end{itemize}

Students will need:
\begin{itemize}
    \item Math Notebook
    \item Pen/pencil
    \item TI-84 calculator
\end{itemize}

\section{Outline}

\begin{itemize}
    \item First draw out what a parabola (an upside down U) is on the white board and ask students what kind of graph it is and what it represents
    \item After hearing a couple responses from students then introduce the concept of what a parabola is and what it is useful for.
    \item Introduce the formula of $ax^2 +bx+c$, to later introduce the quadratic formula.
    \item After grasping the concept of the quadratic formula I will use the  questions that I have from my instruction and we will do it step by step.
    \item They will then work with a partner with three problems that involves using the quadratic formula
    \item After these are work on then we will review and work on another 3 
    \item Give them the worksheet for homework and in class work.
\end{itemize}

\section{Instructions}

\begin{enumerate}
    \item Before the class starts, open the slides (\href{https://docs.google.com/presentation/d/19ciZDfXd51fivbBudBNQqxBPUnCx569onL82jkyxO34/edit#slide=id.g29bd94fe5f5_0_6}{link}) and start playing the playlist of the classroom and check students in for attendance. After the bell rings, begin class.
    \item With continuing this lesson you will use the \textbf{direct-instruction approach} where you will take control of the classroom and lecture the students. The lesson will be structured as a 20 minute instruction, 5-7 minutes of the students trying a problem then transitioning back to a 20 minute instruction to finish it up with 10 minutes of \textbf{seatwork} problems which will also be homework.
    \item Switch to the next slide and ask the students “What is this”, and allow the students to raise their hands and answer on what a parabola is. If a student answers correctly and says a parabola, enforce \textbf{positive reinforcement} as it will encourage the student to participate again. 
    \item Then introduce how the standard quadratic equation explaining what each coefficient represents with each variable 
    \item Introducing a real life problem of how physical objects or sports are involved with the quadratic will follow \textbf{problem-based learning} can better understand the concept of what $x$ represents.
    \item Then Introduce the quadratic formula and compare the standard equation to what is the quadratic formula to show how each variable is represented 
    \item The students will partner up and try to do the first three problems where the solutions will be whole numbers or fractions and not be as challenging.
    \item Then solve one of the problems that the students had a confusion on, if the students understand then move on to the next column
    \item Allow the students to work on the problem for 7-10 minutes.
    \item Then work out the problems with the student on white board and ask them questions of placement of each coefficient.
    \item After working out the problems, provide the students with seatwork which will also be their homework for the night.
\end{enumerate}