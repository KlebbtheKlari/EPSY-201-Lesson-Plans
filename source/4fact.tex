\chapter{Polynomial Factoring I}

\emph{By Nathan Diaz}

\section{Overview}

\subsection{Objectives}

\begin{itemize}
    \item Students will be able to learn how to factor polynomial equations with different value coefficients
    \item Students able to identify coefficients that are associated with variables  
    \item Students able to factor polynomials naturally
\end{itemize}

\subsection{Duration}

\begin{itemize}
    \item 2 minutes - Ask students what is the best possible way to solve the x for a specific polynomial 
    \item 20 minutes - lecturing through writing examples and working through a problem by doing it step by step
    \item 25 minutes - collaboration within small group 
\end{itemize}

\subsection{Materials}

Teachers will need:
\begin{itemize}
    \item White board 
    \item Dry erase marker 
\end{itemize}

Students will need:
\begin{itemize}
    \item Math Notebook
    \item Pen/pencil
    \item TI-84 calculator
\end{itemize}

\section{Outline}

\begin{itemize}
    \item First write a polynomial equation on the board and allow students to solve it without any guidance
    \item Then ask students to explain their way of solving
    \item Increase the challenge by changing the exponents and different coefficient and allow them to solve the problems and solve the step by doing it step by step.
    \item After multiple examples of this, allow them to work on the homework
\end{itemize}

\section{Instructions}

\begin{outline}
    \1 Before the class starts, write the equation beginning the playlist of the classroom and check students in for attendance. After the bell rings, begin class.
    \1 When students are seated and have their notebooks, pencils and calculators, they will be given 3-5 minutes on how to solve a quadratic equation in order to be exposed to seeing exponents and being able to find the values. With this \textbf{mastery learning} we will be able to continue to a new topic.
    \1 With this we transition to involving new equations such as $6x^2+2x+8$ where we allow the student to try to solve the equation.
    \1 After the students haves solve the problem, we will introduce them to special factorizations: $x^2+2xy+y^2 = (x+y)^2$ and $x^2-y^2 = (x-y)(x+y)$.
    \1 With this, we expect students to have some \textbf{prior knowledge} on basic polynomial arithmetic. 
    \1 I transition to \textbf{direct instruction} and teach the student on how to factor polynomials
        \2 Introducing the Greatest Common Factor (G.C.F)
        \2 Along with this we will review how to add and subtract polynomials
    \1 We then transition to the students learning on a \textbf{learner centered approach}. This also be implemented with \textbf{cooperative learning} where students can work on problems together.
    \1 The student will partner up to solve problems on two worksheets.
        \2 \href{https://mathmonks.com/wp-content/uploads/2022/02/College-Algebra-Factoring-Polynomials-Worksheet.jpg}{Link 1}
        \2 \href{https://mathmonks.com/wp-content/uploads/2022/02/Factoring-Out-GCF-of-Polynomials-Worksheet.jpg}{Link 2}
\end{outline}