\chapter{Polynomial GCF and Grouping}

\emph{By Michael Feng}

\section{Overview}

\subsection{Objectives}

Students will be able to:
\begin{itemize}
    \item Define and differentiate between GCF and the Grouping method
    \item Demonstrate and apply factoring using GCF
    \item Demonstrate factorization using the Grouping method
\end{itemize}

\subsection{Materials}

Teacher will need:
\begin{itemize}
    \item Whiteboard or Smartboard \& Projector with document camera
    \item Dry erase markers \& eraser or pen \& paper for document camera
    \item Scratch paper for students
    \item Worksheet - print copies for students (\href{https://www.celinaschools.org/Downloads/Factoring%20By%20Grouping%20Review.pdf}{Link})
\end{itemize}

Students will need:
\begin{itemize}
    \item Pencil/pen
    \item Eraser/whiteout
    \item Readiness to learn
\end{itemize}

\subsection{Duration}

The lesson as a whole will take approximately 50 minutes.
\begin{outline}
    \1 Ice breaker - students will form groups of 3-4 and play 2 truths and a lie (5 minutes)
    \1 Teaching polynomial GCF (15 minutes)
    \1 Teaching factoring with grouping (15 minutes)
    \1 Teacher will then allow the rest of the time for students to work on the given worksheet in their groups of 4. Anything not finished will be homework and the teacher will be walking around during this time (15 minutes)
\end{outline}

\section{Outline}

For the first half of the lesson (factoring using GCF):
\begin{outline}
    \1 Begin with asking the class if they are familiar with the term ``GCF or Greatest Common Factor''
        \2 It is expected that the student's have no idea what the term is but if they do, good on them and I'll give them the opportunity to explain it
    \1 Give the students the definition of GCF
    \1 Present a problem on the Smartboard or Whiteboard and meticulously go through the problem, applying GCF and making sure the students understand it
\end{outline}

For the second half of the lesson (factoring using the grouping method):
\begin{outline}
    \1 After learning what the GCF is, build on top of this by introducing and explaining the Grouping method
    \1 Present a different problem and give students a minute to see if they can solve this
        \2 They shouldn't be able to but you should be willing to allow people to answer if they want to
    \1 Showcase how to solve the different problem, showing that you must first group the different terms, then use GCF, and then factor out the common factor
\end{outline}

For the rest of the class, handout worksheet that has another example with explanation + solution as well as practice problems so students can better understand the topic if they are still confused and make them work on the practice problems together as the exit slip (no homework).

\section{Instructions}

\begin{outline}
    \1 Before class begins, please get accustomed to using the smartboard and the document camera as you will be presenting/teaching to the class. If you are unable to work the camera, no worries! The white board is there as an alternative
        \2 Please play whatever music you wish as long as it is school appropriate and not too loud that it'll distract students and neighboring classes.
        \2 In addition, please make sure to print out today's worksheet and preview it. This will help you get an idea on today's lesson as well as can be used as reference during the lesson.
    \1 Take attendance. Then, please put the students into groups of 3 or 4 (you may choose whatever method to group the students but I recommend numbering them off) and have the students move to their respective groups for the day. After, please let the students know that they will be playing 2 truths and 1 lie as their ice breaker for the day and let them begin (feel free to let the students know that they could also play with you as well)
        \2 2 truths and 1 lie is a game in which one group member tells 2 truths about themself and 1 lie and the other members have to decide collectively what could be the lie. 
        \2 In addition, feel free to time or check the clock as to how long the students are spending on this ice breaker, it should last roughly 5 minutes
        \2 Meanwhile, set up for the lesson for the day. On the document camera setup or whiteboard write: “GCF or Greatest Common Factor”
            \3 When it comes to my ice breakers, I always like to find a ice breaker that utilizes a student's \textbf{creativity} (which is the ability to think about something in a inventive and unique way, often creating innovative solutions to problems) as it not only builds on their critical thinking skill but allows them to be open about themselves and makes them see what sort of ideas that they can come up with on the spot. 
    \1 After the ice breaker activity, please begin today's lesson. Begin with asking the class, “Does anyone know the term Greatest Common Factor, GCF for short?”
        \2 It is expected that many will not know the answer but if any are willing to answer please allow them to.
    \1 After the students have been given the opportunity to guess, please say as well as write out the definition: “it is the largest number (and or) variable that can be evenly divided from a set of two or more numbers.”
        \2 When it comes to my lessons, I tend to use \textbf{Direct instruction} which is a teacher-centered instructional technique in which I, the teacher, aim to guide, direct, and control the students such that they can effectively learn and apply the new material. 
    \1 You will now present an example/problem to the students to get them accustomed to what GCF is. Please write out the problem: $2x^3+6x^2+10x$ and ask the students, using the definition that they have just learned, to find the GCF of the terms.
        \2 Please give them a couple of minutes, the answer should be: $2x$
    \1 If a student has gotten the answer please allow them to explain their thought process and ask the class as whole if they understand what GCF is. Even if there are no questions please reiterate the steps in order to find the GCF.	
        \2 Step 1: When finding the GCF, begin by looking at the variable’s in the polynomial, see if there is a consistent variable in every term. As we can see here, there is a consistent $x$ variable in each term. It is solely just $x$ as the third term only has $x$ to the first degree meaning you can only have $x$ as the largest since you can not take a $x^2$ or $x^3$ from the $10x$ as it is impossible. 
        \2 Step 2: After finding the greatest common variable, please look at the coefficient in each term. As we can see, it is $2$, $6$, and $10$. This means that the greatest common coefficient is $2$ because it is “the largest number that can be evenly divided from this set of coefficients 
        \2 Step 3: Combine the coefficient and term as these are both the most common in all of the terms resulting in a GCF of $2x$
            \3 This type of lesson incorporates \textbf{observational learning} (which is a fairly common learning method in which students acquire the necessary skills and strategies through observing or seeing these skills/strategies being performed and utilized). This is important when it comes to learning new subjects as students oftentimes want to visually see what the subject is and how to perform the necessary steps before trying it out 
            \3 In addition, when it comes to all of my lessons, I would like to have my students' \textbf{sustained attention} (which is the ability to focus their attention for a long period of time) as this material is important for them as they reach upper level courses as well as exams and just showing respect and courtesy to an adult. 
    \1 After this, the students should have a stronger or better idea of what GCF is. Now, you will introduce the Grouping method. Please write out ``Grouping method'' and define it as: ``a specific technique used to factor polynomials''
    \1 Next, write out the equation $3x^2+2x+12x+8$ and proceed to follow and explain these steps out loud to show the students how to perform the Grouping method
        \2 Step 1: Begin by grouping the terms in pairs of two. The grouping should have the first term with the second term and the third term with the fourth term. It should look like this: $(3x^2+2x)+(12x+8)$
        \2 Step 2: Next, after just learning the term GCF, please find the GCF in both of these groups and factor them out. It should look like this: $x(3x+2)+4(3x+2)$
        \2 Step 3: Next, we can see that in both of these groups there is a $(3x+2)$ which means that we can actually factor out the $(3x+2)$ as it is prominent in both terms, leaving us with the final result of: $(3x+2)(x+4)$
        \2 Students may be confused on how the $(x+4)$ came about but it is from the factoring of the $(3x+2)$ and how the $x$ and $4$ where being multiplied by $(3x+2)$, resulting in $(3x+2)(x+4)$
    \1 After completing the lesson for the day, there should be roughly 15 minutes left of class. Please hand out the worksheet to class and let the students work on it together in their groups from today's icebreaker activity. It is strongly encouraged to let students work on the practice problems of the worksheet together. There are additional examples with solutions to aid them if they are still confused with the topic. This will not be for a grade but rather to see if the students understand the topic after today's activity (please do not let them know, to ensure maximum work productivity, let the students know that this will be their exit slip and to finish as many problems as possible. Also, please tell them to staple their scratch work, if any, to their worksheet as I would like to see their thought process when trying to figure out the solution)
    \1 After class ends, remind the class to have a great day and that there will be no homework for today. 
\end{outline}