\chapter{Long Division \& Euclidean Division}

\emph{By Michael Feng}

\section{Overview}

\subsection{Objectives}

Students will be able to:
\begin{itemize}
    \item Define and differentiate Long division and Euclidean Division
    \item Effectively demonstrate long division with polynomials
    \item Demonstrate Euclidean division with polynomials
\end{itemize}

\subsection{Materials}

Teacher will need:
\begin{itemize}
    \item Whiteboard or Smartboard \& Projector with document camera
    \item Dry erase markers \& eraser or pen \& paper for document camera
    \item Worksheet - print copies for students (\href{https://www.hunter.cuny.edu/dolciani/pdf_files/brushup-materials/dividing-polynomials-using-long-division.pdf}{Link})
    \item Computer \& Projector (if using whiteboard)
\end{itemize}

Students will need:
\begin{itemize}
    \item Pencil/pen
    \item Eraser/whiteout
    \item Readiness to learn
\end{itemize}

\subsection{Duration}

The lesson as a whole will take approximately 50 minutes.
\begin{outline}
    \1 Ice breaker - students will form groups of 3-4 and play pictionary with the teacher. The teacher will give each group 1 minute to guess an array of drawings. The highest scoring group will get candy (5 minutes)
    \1 Teaching Long division (15 minutes)
    \1 Teaching Euclidean Division (15 minutes)
    \1 Teacher will then allow the rest of the time for students to work on the given worksheet in their groups of 4. Anything not finished will be homework and the teacher will be walking around during this time (15 minutes)
\end{outline}

\section{Outline}

For the first half of the lesson (Long division with polynomials):
\begin{outline}
    \1 Begin with asking the class if they are familiar with the term “Long division”
        \2 It is expected that the student’s have AN idea what the term is AND should be able to explain it as it was something they were taught but if unknown it is okay
    \1 Review what Long division is, defining the term and then explain how this is applicable to polynomials
    \1 Present a problem on either the Smartboard or Whiteboard and meticulously go through the problem, explaining to students the process of long division between two polynomials
\end{outline}

For the second half of the lesson (Euclidean division with polynomials):
\begin{outline}
    \1 After learning/teaching the process of long division with polynomials, build on top of this by introducing and explaining the Euclidean Division with polynomials
    \1 Present the same problem with the solution and give students a minute to see if they can solve use Euclidean division to check if their work is correct in arriving to the dividend
\end{outline}

For the rest of the class, handout worksheet that has another example with explanation + solution as well as practice problems so students can better understand the topic if they are still confused and make them work on the practice problems together as the exit slip (no homework).

\section{Instructions}

\begin{outline}
    \1 Before class begins, please get accustomed to using the smartboard and the document camera as you will be presenting/teaching to the class. If you are unable to work the camera, no worries! The white board is there as an alternative
        \2 Please play whatever music you wish as long as it is school appropriate and is not too loud that it'll distract from the students and neighboring classes learning 
        \2 In addition, please make sure to print out today's worksheet and preview it. This will help you get an idea on today's lesson as well as can be used as reference during the lesson.
    \1 Take attendance. Then, please put the students into groups of 3 or 4 (you may choose whatever method to group the students but I recommend numbering them off) and have the students move to their respective groups for the day. After, please let the students know that they will be playing pictionary with you as their ice breaker for the day and let them get ready.
        \2 Pictionary is a game in which you will be drawing an array of words on white board and letting one group guess what you are drawing. For every picture they get correct, the group will accumulate one point, most points out all of the groups will be receiving candy from my desk. 
            \3 You may use any word as long as they are appropriate, I like to come up with them on  the spot so as to not allow any groups to have an idea of what I will be drawing. 
            \3 Please spend roughly around one minute per group.
    \1 After the ice breaker activity, please begin today's lesson. Begin with asking the class, “Does anyone know what Long division is? Does anyone remember hearing Long division before?”
        \2 It is expected that many will know the answer, and you should allow them to answer. However if the students do not remember that is okay
    \1 After the students have been given the opportunity to guess, please say as well as write out the definition: “it is the mathematical method for dividing large numbers into smaller groups or parts”. You can then show to the students what it is by clicking the \href{https://www.houseofmath.com/encyclopedia/numbers-and-quantities/arithmetic/division/how-to-do-long-division}{link here} and scrolling to the end of the page. Next you will then explain how this is applicable to polynomials by saying: “this is applicable to polynomials because you can divide polynomials between each other to get their greatest common polynomial between the two and/or a possible remainder”
    \1 You will now present an example/problem to the students to get them accustomed to what Long division between polynomials is. Please write out the problem (in long division form): $(6x^2+7x-20)/(2x+5)$ and ask the students, using the definition that they have just learned as well as their knowledge of division, to try to find the answer or “quotient”
        \2 Please give them a couple of minutes, the answer should be: $3x-4$
        \2 I like to use \textbf{transfer} (which is the process of applying previously learned knowledge and experiences to a new situation) in my lessons as not only do I like to continually reiterate previous lessons but, it involves \textbf{critical thinking} (which is the act of thinking reflectively and productively) which is such an important life skill as it'll make my students accustomed to difficulties and setbacks and will allow to them to be better thinkers and come up with solutions more quickly. 
    \1 If a student has gotten the answer please allow them to explain their thought process. Though it is expected that the students may still be confused as to how to perform long division with polynomials, as such please reiterate these steps as to how to solve any long division polynomial.	
        \2 Step 1: The first step when performing long division is to set up the Long division. First, draw a vertical line. Then, draw a horizontal line beginning at the top of the vertical and pointing towards the right, it does not need to be fairly long. It should look like the letter “L” rotated 90 degrees clockwise. Next, write the $(2x+5)$ on the outside of the “L”. Lastly, put the $(6x^2+7x-20)$ on the inside of the “L”. 
        \2 Step 2: When performing the long division, we first look at the firm terms for each polynomial on both sides. We need to consider how many times can the first term on the outside go in on the first term on the inside. Here, we can see that $2x$ can go into $6x^2$ a total of $3x$ times as $3x * 2x$ yields $6x^2$. 
        \2 Next, because we found that $2x$ can go into $6x^2$ a total of $3x$ times, we put the $3x$ on the line that is above the $6x^2$ and then put $6x^2+15x$ below t he $(6x^2+7x-20)$. How we got this was that because we found that $2x$ can go into $6x^2$ a total of $3x$, we must multiply that $3x$ with the $+5$ that was part of the $2x$ such that we can yield $6x^2$  and be able to cancel out both $6x^2$.
        \2 Step 4: Now we subtract the two polynomials, anything that is inside the “L” will always be subtracted. We should now yield a $-8x-20$, we get the $-8x$ from subtracting the two polynomials and then we drop the $-20$ down as that is a term that we still need to consider. 
        \2 Step 5: Now we repeat step 3 in which we have to determine how many times can $(2x)$ go into $-8x$. Here many will be confused as to why it is not $4$, that is because we are still subtracting the two polynomials and as such if we did $4$, we would have to do $-8x - 8x$ which would be incorrect. Instead we need $-4$ because the negatives will cancel out and yield a positive resulting in: $-8x + 8x$ or net zero. 
        \2 Step 6: After finding the term, place it on the same line where you placed the $3x$. You will then subtract the two polynomials, remember to multiply the  $+5$ with the $-4$ as well. As a result of this final multiplication and subtraction, we have yielded a result of zero. This means that we have fully divided the two polynomials and are left with the answer that is on the line which is, $3x-4$
        \2 Step 7: Tell the class that if you are given a problem in which you cannot divide the leftover term anymore with your given polynomial, that is considered your remainder and represents your leftover amount after fully dividing between the two polynomials. 
            \3 My lessons often incorporate \textbf{encoding} (which is the process of how information gets into a student's memory) through letting the students focus their attention on me and the practice they will be doing or through their work as that constant repetition and practice will encode into their minds
    \1 After this, the students should have a stronger or better idea of how to perform Long division with polynomials. Now, you will build on top of this concept by introducing Euclidean division with polynomials. You will begin by elaborating and writing out what Euclidean division is. 
    \begin{definition}
        If $a$ and $b$ are polynomials, then there exist unique polynomials $q$ and $r$ with $0 \le \deg r < \deg b$ such that $a = bq+r$.
    \end{definition}
    This is a backwards process in order to double check if your work is correct or if you are given a problem in which you have been given all of the elements but one.

    \1 Next, you will use the same question that you gave to the students for long division but ask the students to utilize the formula and all the results they have found to see if the work matches and that they have the write answer. After giving the students some time to utilize the formula, showcase how to use it:
        \2 Step 1: Begin by writing out the formula $a = bq + r$.
        \2 Step 2: Utilize the answers and what we found and plug the variables into the formula. We should have a result of $6x^2+7x-20 = (2x+5)(3x-4) + 0$
            \3 Here the $r$ is zero because we found no remainder from our calculations and as such that means $r$ is zero. 
        \2 Step 3: Using distributive property, distribute the $2x$ into the $3x-4$ and then the $5$ into the $3x-4$. We should be left with: $6x^2-8x+15x-20$.
        \2 Step 4: Combine like terms and check if it matches with our dividend
            \3 Final result of $6x^2+7x-20$ and checking it with our formula we get: $ 6x^2+7x-20 =  6x^2+7x-20 $ which is correct! 
    \1 After completing the lesson for the day, there should be roughly 15 minutes left of class. Please hand out the worksheet to class and let the students work on it together in their groups from today's icebreaker activity. It is strongly encouraged to let students work on the practice problems of the worksheet together. There are additional examples with solutions to aid them if they are still confused with the topic. This will not be for a grade but rather to see if the students understand the topic after today's activity (please do not let them know, to ensure maximum work productivity, let the students know that this will be their exit slip and to finish as many problems as possible. In addition, please let the students know that they must use Euclidean division to check their work and write out their steps as well)
        \2 I always like to utilize \textbf{cooperative learning} (which is the process of learning through the guidance and help of peers in a group setting) when it comes to my classroom as I believe that working with your fellow peers will allow you to not only make friends but also be more comfortable with asking questions and also to learn as everyone has different experiences and methods. 
    \1 After class ends, remind the class to have a great day and that there will be no homework for today. 
\end{outline}