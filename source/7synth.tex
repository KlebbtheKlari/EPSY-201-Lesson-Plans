\chapter{Synthetic Division}

\emph{By Faith Conopeotis}

\section{Overview}

\subsection{Objectives}

Students will be able to:
\begin{itemize}
    \item Define synthetic division
    \item Recognize when to divide polynomials using synthetic division
    \item Learn how to divide polynomials using synthetic division
\end{itemize}

\subsection{Duration}

This lesson will be split up into three parts (roughly 45 min):
\begin{itemize}
    \item Lecture: 25 min
    \item Group work: 10 min
    \item Individual work: 10 min
\end{itemize}

\subsection{Materials}

Teacher will need:
\begin{itemize}
    \item Computer \& Projector
    \item Whiteboard \& Markers
    \item Worksheet (\href{https://docs.google.com/document/d/1HKYbJtd42jh1GWfWh5GD5qbosMRKXrE86Zldi-lMgnQ/edit?usp=sharing}{Link})
    \item Worksheet Answer Key (\href{https://docs.google.com/document/d/11MaOdGLrB1454qrrkZYvaxijsIcwRAB8fi_g2UDCRQE/edit?usp=sharing}{Link})
    \item Slideshow (\href{https://docs.google.com/presentation/d/1nxi8WIntP3aSjfRn5nGwJmopC85xrW9zZ4EOi68K5cQ/edit?usp=sharing}{Link})
    \item Video (\href{https://youtu.be/7mS5LmJffgw?si=Mkinr-KE5xFuVt75}{Link})
\end{itemize}

Students will need:
\begin{itemize}
    \item Pencil/Pen
    \item Computer
    \item Notebook/Paper
    \item Worksheet (pass out to them)
    \item Calculator (optional)
\end{itemize}

\section{Lesson Instructions}

\begin{outline}
    \1 Before class beings, have the slideshow presented on the screen and the worksheet printed out.
    \1 As the students walk in, have the printed worksheet out for them to pick up before going to their seats
    \1 Before starting the slideshow, have the students take out their notebooks to take notes in during the presentation. Make sure to go at a good pace so that they can write what they want down in enough time. 
    \1 Start the slideshow: We want to take on a \textbf{direct instruction approach} during this lesson. Therefore, we want to present the information to the class and guide the students through instructions. We want to be explicit and explain to the students the content we want to teach them. 
        \2 Explain what today's lesson is and the learning objectives
        \2 Explain what Synthetic Division is and why you use it
        \2 Start going through the steps of synthetic division
            \3 Make sure to ask if anyone has any questions after every step so that everyone is on the same page
        \2 Explain the difference between long division and synthetic division
    \1 Have them pull out their worksheets and start working with the people with them at their assigned seats for ten minutes.
        \2 Set group work is a way of enforcing \textbf{cooperating learning}. It's important at the start of learning for students to build off of each other and work together to work out problems. Whenever a new concept is introduced, cooperative learning is an effective way for students to understand faster with the help of other students. 
        \2 Having the students work on this worksheet will allow \textbf{mastery learning}. The practice problems that have the same quick steps every time, will allow students to master this concept fast after enough problems have been solved.
    \1 After ten minutes, have students direct their attention back to the screen and explain that they are going to watch a video on synthetic division on when the coefficient is greater than 1
        \2 Put the example of $3x + 1$ on the whiteboard to explain what the divisor looks like in this situation
    \1 Explain that they are going to open their computers and watch the attached video in the assignment for the day for ten minutes (Mention taking notes while watching the video)
        \2 Students will use their \textbf{prior knowledge} from the slideshow to comprehend the different situation involving synthetic division
    \newpage
    \1 After ten minutes or when there are a couple minutes left of class, have the class direct their attention back to the screen one last time to explain the homework. Let them know that they need to finish any problems they didn't get to on the worksheet and finish taking notes on the video.  
    \1 At the end of class, collect the worksheets from the students that finished them before you dismiss them.
\end{outline}