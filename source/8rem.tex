\chapter{Remainder and Factor Theorem}

\emph{By Faith Conopeotis}

\section{Overview}

\subsection{Objectives}

Students will be able to:
\begin{itemize}
    \item Evaluate polynomials using the remainder theorem
    \item Utilize the factor theorem to solve a polynomial equation
\end{itemize}

\subsection{Duration}

This lesson will take roughly 45 minutes long.

\subsection{Materials}

Teacher will need:
\begin{itemize}
    \item Computer \& Projector
    \item Whiteboard \& Markers
    \item Worksheet (\href{https://drive.google.com/file/d/1crXwcymstVJxKvmbOWzUk5PI2Api0fNN/view?usp=sharing}{Page 1}) (\href{https://drive.google.com/file/d/1tynle8oNpN0HsN0GIBhGMZQ-xCCrXfoI/view?usp=sharing}{Page 1 KEY}) (\href{https://drive.google.com/file/d/1uWLIYOQuwUsIJVg4Sb2ZV4Kpk6U4Bh_h/view?usp=sharing}{Page 2 with KEY at the bottom})
    \item Slideshow (\href{https://docs.google.com/presentation/d/1H6u-uzQyJrKwW_j5G5LfLVYLoSvyrSMNbyg5xnXpMx0/edit?usp=sharing}{Link})
    \item Video (\href{https://youtu.be/_IPqCaspZOs}{Link}, also in slideshow)
\end{itemize}

Students will need:
\begin{itemize}
    \item Pencil/Pen
    \item Notebook/Paper
    \item Worksheet (pass out to them)
    \item Calculator (optional)
\end{itemize}

\section{Lesson Instructions}

\begin{outline}
    \1 Before class begins, have the slideshow presented on the screen and print out the worksheets. 
    \1 As the students walk in, have the worksheets out for them to pick up before going to their seats
    \1 Before starting the slideshow, have the students take out their notebooks to take notes in during the presentation. Make sure to go at a good pace so that they can write what they want down in enough time. 
    \1 Start presenting the slideshow
        \2 Start off with the warm up to review synthetic division
        \2 Explain the objectives of the day
            \3 The \textbf{performance criteria} should be mentioned to the students during the beginning of the class. It's helpful for the students to understand how it's expected to perform and learn the following concept in order to focus and stay on task
        \2 Explain to the class that they are going to watch this video to learn about these theorems. Then proceed to play the remainder and factor theorem youtube video for the whole class to see on the projector
        \2 After watching the video, review with the class what was talked about in the video and ask if anyone has any questions.
        \2 Then use the whiteboard to write out the problems on slides 7-8 so that you can go through step by step. 
        \2 Go through the example using the remainder theorem
        \2 Go through the example using the factor theorem
    \1 Move on to group work and have them pull out their worksheets. 
        \2 Group work is a good way for students to utilize \textbf{peer tutoring}. It's always beneficial to see students help each other out before the teacher needs to intervene. 
    \1 Explain to them before starting that the rest of the worksheets will be for homework. 
        \2 This \textbf{negative reinforcement} will hopefully motivate the students to finish their work in class so that they don't have to take away time in their night to finish it. Therefore, if the students are on task, they should get everything done. 
    \1 As the students are working in groups, walk around the class to see how your students are doing. It's important to use \textbf{scaffolding} when noticing a student or group of students are struggling. All students work at different paces and have concepts come easier to them. Therefore, it's necessary to figure out where the student/s are struggling and go from there to best help them understand. 
    \1 When there are a couple minutes of class left, direct their attention back to the screen to explain the homework. Let them know that they need to finish any problems they didn't get to on the worksheet. Also, mention that they should reach out if they need help or plan a meeting to help with any part of the homework/class work. 
    \1 At the end of class, collect the worksheets from the students that finished them before you dismiss them.
\end{outline}